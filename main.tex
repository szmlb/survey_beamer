\documentclass[8pt, dvipdfmx]{beamer}

\title{Survey template with beamer}
\author{@szmlb}
\date{\today}

\usetheme{Warsaw}
\usecolortheme{rose}
%\usetheme{Copenhagen} % テーマは指定しなくともよい.

\begin{document}

\maketitle

\section{目次}
\begin{frame}{目次}
 \tableofcontents
\end{frame}

\section{README}
\begin{frame}{README}
  \begin{itemize}
   \item 落合先生のサーベイまとめ方法を踏襲し, レイアウトにこだわることなく内容の記述に集中できるようbeamerでの環境を整備した.
   \item そもそもレイアウトはあまり重要ではないので,  ブログ等で項目別に箇条書きするだけでもいい気がする.
   \item とりあえずbeamerの自分用メモ代わりとして公開.
  \end{itemize}
\end{frame}

\section{論文レビュー}
\subsection{Sampling-based approach}
\begin{frame}{RRT-connect \cite{RRT-connect}}
    \begin{columns}[t]
        \begin{column}{0.5\textwidth}
          \begin{block}{どんなもの?}
            Rapidly-exploring Rondamized Trees: RRTを始点と終点から同時に実行して, 二つの探索木を簡単なgreedy heuristicによって接続する.
          \end{block}
          \begin{block}{先行研究と比べてどこがすごい?}
           前処理が要らず, incrementalな実行で, 難しい問題に対して解が得られている.
          \end{block}
          \begin{block}{技術や手法のキモはどこ?}
            greedyな探索と確率的な探索によるバランスの取れた探索. probabilistically completeであることが証明されている.
          \end{block}
        \end{column}
        \begin{column}{0.5\textwidth}
          \begin{block}{どうやって有効だと検証した?}
            二次元の経路計画問題やpiano mover's problem, 6自由度マニピュレータの経路計画, ヒトの腕(7自由度)の経路計画にRRT-connectを利用して, 問題が解けることを示した.
          \end{block}
          \begin{block}{議論はある?}
            RRTのノード更新ステップ幅の最適化. CONNECTアルゴリズムの改善. 最近傍点探索手法の検討. incrementalな衝突検知.
          \end{block}
          \begin{block}{次に読むべき論文は?}
            Probabilistic Road Map: PRM[15]やRRTの原著論文[18].
          \end{block}
        \end{column}
    \end{columns}
\end{frame}

\begin{frame}{手法・実験環境の詳細}
論文の特徴的な式やアルゴリズム, 図等のまとめるがあると後で見返す時に良いかも.
\end{frame}

\section{参考文献}
\begin{frame}{References}

\begin{thebibliography}{99}
%\setlength{\itemsep}{-.5zw}
\beamertemplatetextbibitems

\bibitem{RRT-connect}
J. J. Kuffner and S. M. LaValle, ``RRT-connect: An Efficient Approach to Single-query Path Planning," in Proc. IEEE Int. Conf. on Robotics and Automation, pp. 995-1001, 2000

\end{thebibliography}

\end{frame}

\end{document}
